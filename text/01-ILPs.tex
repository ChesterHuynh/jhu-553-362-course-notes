\section{(Binary) Integer Linear Programming}

\subsection{Review of Linear Programming}
\label{subsec:review-of-linear-programming}

Recall that linear programs are a class of problems that have been well studied and can be solved quickly (in terms of asymptotics, this means they can be solved in polynomial time).

\begin{defn}[Linear program]
\label{defn:linear-program}
    Given $A \in \R^{m\times n}$, $b \in \R^m$, and $c \in \R^n$, a \textit{linear program} can be written in either standard form or canonical (also called symmetric) form as follows:
    \begin{multicols}{2}
        \noindent
        \begin{equation*}
        \begin{array}{lllr@{}l@{}l@{}}
        \text{standard form} 
            & (LP)  & \min  & c^T x &           \\
            &       & \st   & Ax    & {}= b     \\
            &       &       &  x    & {}\ge \Vec{0}
        \end{array}
        \end{equation*}
    
        \noindent
        \begin{equation*}
        \begin{array}{lllr@{}l@{}l@{}}
        \text{canonical form} 
            & (LP)  & \min  & c^T x &           \\
            &       & \st   & Ax    & {}\ge b   \\
            &       &       &  x    & {}\ge \Vec{0}.
        \end{array}
        \end{equation*}
    \end{multicols}
\end{defn}

In optimization, a given problem has a dual that is related to its primal form. In linear programming, the duality gap is 0 if and only if the optimal solution is feasible for both problems, which is helpful if one is easier to solve than the other.

\begin{defn}[Dual program]
    \label{defn:dual-program}
    Given $A \in \R^{m \times n}$, $b \in \R^m$, and $c \in \R^n$, a \textit{dual program} can be written in either standard form or canonical (also called symmetric) form as follows:
    \begin{multicols}{2}
        \noindent
        \begin{equation*}
        \begin{array}{lllr@{}c@{}}
        \text{standard form} 
            & (DP)  & \max  & b^T y &           \\
            &       & \st   & A^T y & {}\le c   \\
        \end{array}
        \end{equation*}
        
        \noindent
        \begin{equation*}
        \begin{array}{lllr@{}c@{}}
        \text{canonical form} 
            & (DP)  & \max  & b^T y &           \\
            &       & \st   & A^T y & {}\le c   \\
            &       &       &  y    & {}\ge \Vec{0}.
        \end{array}
        \end{equation*}
        \end{multicols}
\end{defn}

The notation used above will also be used throughout unless otherwise specified. Additionally, $ofv_P$ will denote the objective function value for a particular problem $(P)$ and $oofv_P$ will denote the optimal objective function value for a particular problem $(P)$. The subscript will be dropped when the problem of interest is clear.

We recall some immediate results that relate a linear program $(LP)$ to its dual program $(DP)$.

\begin{thm}[Weak duality for LPs]
    \label{thm:weak-duality-LP}
    If $x$ is feasible in $(LP)$ and $y$ is feasible in $(DP)$, then $ofv_{DP}(y) \le ofv_{LP}(x)$.
\end{thm}

\begin{thm}[Supervisor principle for LPs]
    \label{thm:supervisor-principle-LP}
    If $x$ is feasible in $(LP)$, $y$ is feasible in $(DP)$, and $ofv_{DP}(y) = ofv_{LP}(x)$, then $x$ is optimal in $(LP)$ and $y$ is optimal in $(DP)$.
\end{thm}

\begin{thm}[Strong duality for LPs]
    \label{thm:strong-duality-LP}
    If $(LP)$ is feasible and $(DP)$ is feasible, then there exists $x$ that is feasible in $(LP)$ and $y$ that is feasible in $(DP)$ such that $ofv_{DP}(y) = ofv_{LP}(x)$. Hence, $x$ is optimal in $(LP)$ and $y$ is optimal in $(DP)$.
\end{thm}


\subsection{Introduction of (Binary) Integer Linear Programs}
\label{subsec:introduction-of-ilps}

In general, solving integer linear/dual programs is NP-hard.

\begin{defn}[Integer linear/dual program]
    \label{def:integer-linear/dual-program}
    Given $A \in \R^{m \times n}$, $b \in \R^m$, and $c \in \R^n$, a (binary) integer linear program $(ILP)$ and its integer dual program $(IDP)$ in canonical form is written as
    \begin{multicols}{2}
        \noindent
        \begin{equation*}
        \begin{array}{llr@{}l@{}}
            (ILP)   & \min  & c^T x &                   \\
                    & \st   & Ax    & {}\ge b           \\
                    &       &  x    & {}\in \{0, 1\}^n
        \end{array}
        \end{equation*}
        
        \noindent
        \begin{equation*}
        \begin{array}{llr@{}l@{}}
            (IDP)   & \min  & b^T y &                   \\
                    & \st   & A^T y & {}\le c           \\
                    &       &  y    & {}\in \{0, 1\}^n.
        \end{array}
        \end{equation*}
    \end{multicols}
\end{defn}

\begin{rmk}
    \label{rmk:ilp-lp-inequalities}
    Since the feasible region for the $(ILP)$ is a subset of the feasible region for the $(LP)$ (with all else same), then $oofv_{LP} \le oofv_{ILP}$. By a similar argument, $oofv_{IDP} \le oofv_{DP}$. This gives the following relations, assuming existence of respective solutions,
    \[
        oofv_{IDP} \le oofv_{DP} = oofv_{LP} \le oofv_{ILP}.
    \]
\end{rmk}

The duality gap between a given $(ILP)$ and $(IDP)$ does not have strong guarantees like for linear programs. However, we do have the following results relating $(ILP)$ and $(IDP)$.

\begin{thm}[Weak duality for ILPs]
    \label{thm:weak-duality-ILP}
    If $x$ is feasible in $(ILP)$ and $y$ is feasible in $(IDP)$, then $ofv_{IDP}(y) \le ofv_{ILP}(x)$.
    
    \begin{proof}
        Since $x$ is feasible in the relaxation $(LP)$ of $(ILP)$ and $y$ is feasible in the relaxation $(DP)$ of $(IDP)$, by weak duality for LPs, $ofv_{IDP}(y) \le ofv_{DP}(y) \le ofv_{LP}(x) \le ofv_{ILP}(x)$.
    \end{proof}
\end{thm}

\begin{thm}[Supervisor principle for ILPs]
    \label{thm:supervisor-principle-ILP}
    If $x$ is feasible in $(ILP)$, $y$ is feasible in $(IDP)$, and $ofv_{IDP}(y) = ofv_{ILP}(x)$, then $x$ is optimal in $(ILP)$ and $y$ is optimal in $(IDP)$.
    
    \begin{proof}
        By weak duality for ILPs, $ofv_{ILP}$ is lower bounded by $ofv_{IDP}(y)$ for all $y$ feasible in $(IDP)$, so $x$ is optimal in $(ILP)$. By a similar argument, $y$ is optimal in $(IDP)$. Further, by a squeeze argument, $oofv_{IDP}(y) = oofv_{DP}(y) = oofv{LP}(x) = oofv_{ILP}(x)$.
    \end{proof}
\end{thm}

In general, we do not have an analogous strong duality relation between a given $(ILP)$ and $(IDP)$, so there could be a duality gap.

\begin{exm}
    Let $A = [2]$, $b = [1]$, and $c = [1]$. Then,
    \begin{multicols}{2}
        \noindent
        \begin{equation*}
        \begin{array}{llr@{}l@{}}
            (ILP)   & \min  &  x    &               \\
                    & \st   & 2x    & {}\ge 1       \\
                    &       &  x    & {}\in \{0, 1\}
        \end{array}
        \end{equation*}
        
        \noindent
        \begin{equation*}
        \begin{array}{llr@{}l@{}}
            (IDP)   & \max  &  y    &                   \\
                    & \st   & 2y    & {}\le 1           \\
                    &       &  y    & {}\in \{0, 1\}.
        \end{array}
        \end{equation*}
    \end{multicols}
    $oofv_{ILP} = 1$ since $x = 1$ is the only feasible solution for $(ILP)$. $oofv_{IDP} = 0$ since $y = 0$ is the only feasible solution for $(IDP)$. This gives a duality gap of 1. 
    
    However, the linear relaxations have optimal solutions $x = y = \frac12$ and $oofv_{LP} = oofv_{DP} = \frac12$, upholding strong duality for LPs.
\end{exm}

Since solving ILPs is NP-Hard in general, we can instead solve the LP relaxation in poly-time. If we obtain an optimal solution $x^*$ for the $LP$, we know the following about the ILP:
\begin{itemize}
    \item $oofv_{LP} \le oofv_{ILP}$
    \item if $x^*$ happens to be binary-valued, then $x^*$ is also optimal in ILP.
\end{itemize}

\begin{defn}[Dual problems]
    \label{def:dual-problems}
    Consider the problems
    \begin{multicols}{2}
        \noindent
        \begin{equation*}
        \begin{array}{lll@{}}
            (P)     & \min  &  f(x)     \\
                    & \st   &   x \in S \\
        \end{array}
        \end{equation*}
        
        \noindent
        \begin{equation*}
        \begin{array}{lll@{}}
            (Q)     & \min  &  g(y)     \\
                    & \st   &   y \in T. \\
        \end{array}
        \end{equation*}
    \end{multicols}
    If for all $x \in S$ and $y \in T$, $g(y) \le f(x)$, then we say $(P)$ and $(Q)$ are \textit{dual problems}.
\end{defn}


\subsection{Graph Theory}
\label{subsec:graph-theory}

\begin{defn}[Simple graph]
    \label{def:simple-graph}
    A \textit{simple graph} $G = (V, E)$ consists of 
    \begin{itemize}
        \item a set $V$ of vertices
        \item a set $E$ of 2-element subsets of $V$.
    \end{itemize}
    This definition does not allow for
    \begin{itemize}
        \item self-loops
        \item parallel edges
        \item directed edges
    \end{itemize}
\end{defn}

\begin{defn}[Matching]
    \label{def:matching}
    $F \sse E$ is a \textit{matching} if no 2 members of $F$ share endpoints. Note, $\emptyset$ is vacuously a matching. The \textit{matching number} $\alpha'(G)$ for a graph $G$ is
    \[
        \alpha'(G) \defeq \max_{F \sse E \text{ matching}} |F|.
    \]
    A \textit{perfect matching} is a matching that uses all vertices.
\end{defn}

\begin{defn}[Independent set]
    \label{def:independent-set}
    $S \sse V$ is an \textit{independent set} if no 2 members of $S$ are adjacent. Note, $\emptyset$ is vacuously an independent set. The \textit{independence number} $\alpha(G)$ for a graph $G$ is
    \[
        \alpha(G) \defeq \max_{S \sse V \text{ ind set}} |S|.
    \]
\end{defn}

\begin{defn}[Clique]
    \label{def:clique}
    $S \sse V$ is a \textit{clique} if every pair of vertices in $S$ are adjacent. Note, $\emptyset$ is vacuously a clique. The \textit{clique number} $\omega(G)$ for a graph $G$ is
    \[
        \omega(G) \defeq \max_{S \sse V \text{ clique}} |S|.
    \]
\end{defn}

\begin{defn}[Vertex cover]
    \label{def:vertex-cover}
    $S \sse V$ is a \textit{vertex cover} if every edge of $E$ has an endpoint in $S$. The $\textit{vertex cover number}$ $\beta(G)$ for a graph $G$ is
    \[
        \beta(G) \defeq \min_{S \sse V \text{ vertex cover}} |S|.
    \]
\end{defn}

\begin{defn}[Edge cover]
    \label{def:edge-cover}
    $F \sse E$ is a \textit{edge cover} if every vertex of $V$ is an endpoint of an edge in $F$. The $\textit{edge cover number}$ $\beta'(G)$ for a graph $G$ is
    \[
        \beta'(G) \defeq \min_{F \sse E \text{ edge cover}} |F|.
    \]
\end{defn}

\begin{thm}[Weak duality for matchings and vertex covers]
    \label{thm:weak-duality-matchings-vertex-covers}
    For all graphs $G$, $\alpha'(G) \le \beta(G)$.
    
    \begin{proof}
        Let $F \sse E$ be a maximum matching and $S \sse V$ be a minimum vertex cover. For every edge of $F$, there exists an endpoint in $S$ by definition of a vertex cover. Let $M$ be the set of these vertices, which are all distinct by definition of a matching. Thus, $\alpha'(G) \equiv |F| = |M| \le |S| \equiv \beta(G)$.
    \end{proof}
\end{thm}

\begin{defn}[Bipartite graph]
    \label{def:bipartite-graph}
    A graph $G = (V, E)$ is \textit{bipartite} if $V$ can be partitioned into 2 independent sets $X$, $Y$, that is, every edge has an endpoint in $X$ and an endpoint in $Y$.
\end{defn}

\begin{thm}[K\"onig-Egervary]
    \label{thm:konig-egervary}
    If $G$ is bipartite, then $\alpha'(G) = \beta(G)$.
\end{thm}
