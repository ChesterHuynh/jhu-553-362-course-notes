\section{(Binary) Integer Linear Programming}

\subsection{Review of Linear Programming}

Recall that linear programs are a class of problems that have been well studied and can be solved quickly (in terms of asymptotics, this means they can be solved in polynomial time).

\begin{defn}[Linear program]
\label{defn:dual-program}
    Given $A \in \R^{m\times n}$, $b \in \R^m$, and $c \in \R^n$, a \textit{linear program} can be written in either standard form or canonical (also called symmetric) form as follows:
    \begin{multicols}{2}
    \noindent\begin{equation*}
    \begin{array}{lllr@{}c@{}r@{}l}
    \text{standard form} 
        & \text{(LP)}   & \min  & c^T x &           \\
        &               & \st   & Ax    & {}= b     \\
        &               &       &  x    & {}\ge \Vec{0}
    \end{array}
    \end{equation*}
    
    \noindent\begin{equation*}
    \begin{array}{lllr@{}c@{}r@{}l}
    \text{canonical form} 
        & \text{(LP)}   & \min  & c^T x &           \\
        &               & \st   & Ax    & {}\ge b   \\
        &               &       &  x    & {}\ge \Vec{0}.
    \end{array}
    \end{equation*}
    \end{multicols}
\end{defn}
In optimization, a given problem has a dual that is related to its primal form. In linear programming, the duality gap is 0 if and only if the optimal solution is feasible for both problems, which is helpful if one is easier to solve than the other.

\begin{defn}[Dual program]
    \label{defn:dual-program}
    Given $A \in \R^{m \times n}$, $b \in \R^m$, and $c \in \R^n$, a \textit{dual program} can be written in either standard form or canonical (also called symmetric) form as follows:
    \begin{multicols}{2}
    \noindent\begin{equation*}
    \begin{array}{lllr@{}c@{}r@{}l}
    \text{standard form} 
        & \text{(LP)}   & \max  & b^T y &           \\
        &               & \st   & A^T y & {}\le c   \\
    \end{array}
    \end{equation*}
    
    \noindent\begin{equation*}
    \begin{array}{lllr@{}c@{}r@{}l}
    \text{canonical form} 
        & \text{(LP)}   & \max  & b^T y &           \\
        &               & \st   & A^T y & {}\le c   \\
        &               &       &  y    & {}\ge \Vec{0}.
    \end{array}
    \end{equation*}
    \end{multicols}
\subsection{Introduction of ILPs}
\end{defn}

